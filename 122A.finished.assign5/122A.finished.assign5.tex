%   Library imports
\documentclass{article}
\usepackage{enumitem}
\usepackage{listings}
\usepackage{amsfonts}
\usepackage{latexsym}
\usepackage{fullpage}
\usepackage{graphicx}
\usepackage{paralist}
\usepackage{tikz-timing}

\lstdefinelanguage{VHDL}{
  morekeywords={
    library,use,all,ENTITY,IS,PORT,IN,OUT,end,architecture,of,
    begin,and, ARCHITECTURE, IF, THEN, SIGNAL,END, PROCESS
  },
  morecomment=[l]--
}

\usepackage{xcolor}
\colorlet{keyword}{blue!100!black!80}
\colorlet{comment}{green!90!black!90}
\lstdefinestyle{vhdl}{
  language     = VHDL,
  basicstyle   = \ttfamily\scriptsize,
  keywordstyle = \color{keyword}\bfseries\ttfamily,
  commentstyle = \color{comment}\ttfamily,	
  tabsize=1
}

\renewcommand{\lstlistingname}{Code}

%   Margin configuration
%   TA: Default margins are too wide all the way around. Reset them to below.
\setlength{\topmargin}{-.5in}
\setlength{\textheight}{9in}
\setlength{\evensidemargin}{0in}
\setlength{\oddsidemargin}{0in}
\setlength{\textwidth}{6.25in}


%\let\oldenumerate\enumerate
%\renewcommand{\enumerate}{
    %\oldenumerate                            <----------     alz:  Old margins (I presume)
    %\setlength{\itemsep}{1pt}
    %\setlength{\parskip}{0pt}
    %\setlength{\parsep}{0pt}
%}
\def\HS{\hspace{\fontdimen2\font}}


%   The Beginning of the LaTex file content

%       Header is now correctly formatted
%   Beginning of header

\begin{document}

\title{
    \textsc{ Intro to Databases \\ EECS 116} \\ 
    \vspace{2pc} 
    \textbf{ Assignment 5 \\ SQL Constraints }
}

\author{
    Aaron Zhong - 67737879 - alzhong@uci.edu \\
    Tina Li     - 92928656 -  tinal7@uci.edu \\
    Andy Le     - 70829342 -  andyl8@uci.edu \\ \vspace{1pc} \\
    ICS Department                           \\ 
    Donald Bren School of Information and Computer Science   \\ University of California, Irvine
}


\date{February 26, 2016}
\maketitle
\tableofcontents


%   Header ends

%   Assignment content below

\newpage

%   Section 1
\section{Questions}

%    \includegraphics[width=0.8\textwidth]{screenshot.png}

%   Section 2
\section{Answers}

    \subsection{Problem 1.1}

        %%%%%%%%%%%%%%%%%%%%%%%%%%%%%%%%%%%%%%%%%%%%%%%%%%%%%%%%%%%%%%% 

        \subsubsection{Part A}
        \begin{verbatim}

        SELECT AVG(rating), SUM(rating), COUNT(rating), COUNT(*)
    	    FROM seller;
        
        \end{verbatim}

        \centerline{\includegraphics[width=0.5\textwidth]{11a.png}}


        %%%%%%%%%%%%%%%%%%%%%%%%%%%%%%%%%%%%%%%%%%%%%%%%%%%%%%%%%%%%%%% 
        \subsubsection{Part B}


        \begin{verbatim}

        SELECT (SUM(rating) / COUNT(rating)),(SUM(rating) / COUNT(*)), AVG(rating)
    	FROM seller;

        \end{verbatim}

        \centerline{\includegraphics[width=0.5\textwidth]{11b.png}}

        If we divide the sum computed above by the count of the ratings, we would get
        exactly the same result as if we used AVG(). If we divided the sum computed
        above by the count of sellers, then we would have a different average than if we
        used AVG(). This is because COUNT(rating) only counts ratings which are not
        null, while COUNT(*) counts the number of tuples.

        %%%%%%%%%%%%%%%%%%%%%%%%%%%%%%%%%%%%%%%%%%%%%%%%%%%%%%%%%%%%%%% 


        \subsubsection{Part C}
%
        Left outer join of S1 and S2, join condition: S1.sid = S2.sid

        \begin{verbatim}

        SELECT *
        	FROM S1 LEFT OUTER JOIN S2 
            ON s1.sid=s2.sid;

        \end{verbatim}

        \centerline{\includegraphics[width=0.5\textwidth]{11ci.png}}
%
        Right outer join of S1 and S2, join condition: S1.sid = S2.sid 

        \begin{verbatim}

        SELECT *
        	FROM S1 RIGHT OUTER JOIN S2
            ON s1.sid=s2.sid;

        \end{verbatim}
        \centerline{\includegraphics[width=0.5\textwidth]{11cii.png}}
%
        Full outer join of S with itself (seller)

        \begin{verbatim}

        SELECT *
            FROM S LEFT OUTER JOIN Seller
            ON s.sid = seller.sid
        UNION
        SELECT *
        	FROM S RIGHT OUTER JOIN Seller

        \end{verbatim}
        \centerline{\includegraphics[width=0.5\textwidth]{11ciii.png}}


    \subsection{Problem 1.2}
%
        \subsubsection{Part A}
        \begin{verbatim}

        CREATE TABLE Emp(
        	eid int,
            ename varchar(100),
            age int,
        	salary real,
            CHECK (salary < 10000)
            );

        \end{verbatim}
%
        \subsubsection{Part B}
%
        MySQL Version
        
        \begin{verbatim}

        CREATE TABLE Dept(
        	did int,
            budget real,
            managerid int
        );

        ALTER TABLE dept
        ADD CHECK (Dept.managerid IN (SELECT Emp.eid FROM emp, dept 
            WHERE emp.age > 30 AND emp.eid = dept.managerid));

        \end{verbatim}

        SQL (in class version)

        \begin{verbatim}

        CREATE TABLE Dept(
        	did int,
            budget real,
            managerid int,
            CHECK (Dept.managerid IN (SELECT Emp.eid FROM emp, dept 
                WHERE emp.age > 30 AND emp.eid = dept.managerid))
        	);

        \end{verbatim}

        \subsubsection{Part C}
%
        \begin{verbatim}

        CREATE ASSERTION OldManager CHECK
        (NOT EXISTS
        	(SELECT Emp.eid
            FROM emp, dept
            WHERE emp.age < 30 AND emp.eid = dept.managerid));

        \end{verbatim}


        \subsubsection{Part D}
%
        Assertion is better since it is not restricted a single relation that a check
        would normally validate.  Compared to a check, assertions detect changes on
        any/all relations mentioned in the assert and is always guaranteed to hold.

        \subsubsection{Part E}
%
        SQL (in class version)

        \begin{verbatim}

        CREATE TRIGGER NoLowerAge
        AFTER UPDATE ON emp
        REFERENCING
        	OLD AS OldTuple,
            NEW AS NewTuple
        WHEN (OldTuple.age > NewTuple.age)
        	UPDATE emp
            SET age = OldTuple.age
            WHERE name = NewTuple.name
        FOR EACH ROW;

        \end{verbatim}

        Version which works with MySQL

        \begin{verbatim}

        DELIMITER //
        
        DROP TRIGGER IF EXISTS NoLowerAge//
        CREATE TRIGGER NoLowerAge 
        BEFORE UPDATE ON emp
        FOR EACH ROW 
        	BEGIN
        		IF NEW.age < OLD.age THEN
        			SET NEW.age = OLD.age;
        		END IF;
        	END;//
        
        DELIMITER ;

        \end{verbatim}

    \subsection{Problem 1.3}

        \subsubsection{Part A}

        \centerline{\includegraphics[width=0.2\textwidth]{13a.png}}

        \subsubsection{Part B}

        \centerline{\includegraphics[width=0.2\textwidth]{13b.png}}


\end{document}
%   Documents ends
