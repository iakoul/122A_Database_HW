%   Library imports
\documentclass{article}
\usepackage{enumitem}
\usepackage{listings}
\usepackage{amsfonts}
\usepackage{latexsym}
\usepackage{fullpage}
\usepackage{graphicx}
\usepackage{paralist}
\usepackage{tikz-timing}

\lstdefinelanguage{VHDL}{
  morekeywords={
    library,use,all,ENTITY,IS,PORT,IN,OUT,end,architecture,of,
    begin,and, ARCHITECTURE, IF, THEN, SIGNAL,END, PROCESS
  },
  morecomment=[l]--
}

\usepackage{xcolor}
\colorlet{keyword}{blue!100!black!80}
\colorlet{comment}{green!90!black!90}
\lstdefinestyle{vhdl}{
  language     = VHDL,
  basicstyle   = \ttfamily\scriptsize,
  keywordstyle = \color{keyword}\bfseries\ttfamily,
  commentstyle = \color{comment}\ttfamily,	
  tabsize=1
}

\renewcommand{\lstlistingname}{Code}

%   Margin configuration
%   TA: Default margins are too wide all the way around. Reset them to below.
\setlength{\topmargin}{-.5in}
\setlength{\textheight}{9in}
\setlength{\evensidemargin}{0in}
\setlength{\oddsidemargin}{0in}
\setlength{\textwidth}{6.25in}


%\let\oldenumerate\enumerate
%\renewcommand{\enumerate}{
    %\oldenumerate                            <----------     alz:  Old margins (I presume)
    %\setlength{\itemsep}{1pt}
    %\setlength{\parskip}{0pt}
    %\setlength{\parsep}{0pt}
%}
\def\HS{\hspace{\fontdimen2\font}}


%   The Beginning of the LaTex file content

%       Header is now correctly formatted
%   Beginning of header

\begin{document}

\title{
    \textsc{ Intro to Databases \\ EECS 116} \\ 
    \vspace{2pc} 
    \textbf{ Assignment 7 \\ Relational Design, Indexing \& Transactions}
}

\author{
    Aaron Zhong - 67737879 - alzhong@uci.edu \\
    Tina Li     - 92928656 -  tinal7@uci.edu \\
    Andy Le     - 70829342 -  andyl8@uci.edu \\ \vspace{1pc} \\
    ICS Department                           \\ 
    Donald Bren School of Information and Computer Science   \\ University of California, Irvine
}


\date{March 11, 2016}
\maketitle
\tableofcontents


%   Header ends

%   Assignment content below

\newpage

\section{Answers}

%1
    \subsection{Problem 1}
    %1.a
        \subsubsection{Problem 1.a} 

        The minimum total number page reads required for the tree is [1] (if the key is in the root)

        \noindent The maximum total number page reads required for the tree is [3] (if the key is in a leaf)

        \noindent The efficient, quick searching property of B+ trees come in play here.

    %1.b
        \subsubsection{Problem 1.b}

       
        \centerline{\includegraphics[width=1\textwidth]{1b.png}}

    %1.c
        \subsubsection{Problem 1.c}

        \centerline{\includegraphics[width=1\textwidth]{1c.png}}

\end{document}
%   Documents ends
