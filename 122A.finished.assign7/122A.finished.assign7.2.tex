%   Library imports
\documentclass{article}
\usepackage{enumitem}
\usepackage{listings}
\usepackage{amsfonts}
\usepackage{latexsym}
\usepackage{fullpage}
\usepackage{graphicx}
\usepackage{paralist}
\usepackage{tikz-timing}

\lstdefinelanguage{VHDL}{
  morekeywords={
    library,use,all,ENTITY,IS,PORT,IN,OUT,end,architecture,of,
    begin,and, ARCHITECTURE, IF, THEN, SIGNAL,END, PROCESS
  },
  morecomment=[l]--
}

\usepackage{xcolor}
\colorlet{keyword}{blue!100!black!80}
\colorlet{comment}{green!90!black!90}
\lstdefinestyle{vhdl}{
  language     = VHDL,
  basicstyle   = \ttfamily\scriptsize,
  keywordstyle = \color{keyword}\bfseries\ttfamily,
  commentstyle = \color{comment}\ttfamily,	
  tabsize=1
}

\renewcommand{\lstlistingname}{Code}

%   Margin configuration
%   TA: Default margins are too wide all the way around. Reset them to below.
\setlength{\topmargin}{-.5in}
\setlength{\textheight}{9in}
\setlength{\evensidemargin}{0in}
\setlength{\oddsidemargin}{0in}
\setlength{\textwidth}{6.25in}


%\let\oldenumerate\enumerate
%\renewcommand{\enumerate}{
    %\oldenumerate                            <----------     alz:  Old margins (I presume)
    %\setlength{\itemsep}{1pt}
    %\setlength{\parskip}{0pt}
    %\setlength{\parsep}{0pt}
%}
\def\HS{\hspace{\fontdimen2\font}}


%   The Beginning of the LaTex file content

%       Header is now correctly formatted
%   Beginning of header

\begin{document}

\title{
    \textsc{ Intro to Databases \\ EECS 116} \\ 
    \vspace{2pc} 
    \textbf{ Assignment 7 \\ Relational Design, Indexing \& Transactions}
}

\author{
    Aaron Zhong - 67737879 - alzhong@uci.edu \\
    Tina Li     - 92928656 -  tinal7@uci.edu \\
    Andy Le     - 70829342 -  andyl8@uci.edu \\ \vspace{1pc} \\
    ICS Department                           \\ 
    Donald Bren School of Information and Computer Science   \\ University of California, Irvine
}


\date{March 11, 2016}
\maketitle
\tableofcontents


%   Header ends

%   Assignment content below

\newpage

\section{Answers}
%2
    \subsection{Problem 2}
    %2.1
        \subsubsection{Problem 2.1}
        
        \begin{verbatim}
           
            SELECT CITY, COUNT(*)
                FROM user
                GROUP by city;

        \end{verbatim}
    %2.2
        \subsubsection{Problem 2.2}

        \begin{verbatim}

            CREATE INDEX city_index ON user(city);

        \end{verbatim}
            We chose to use city as the index because that is the condition we are looking for and grouping by. 
            This directly increases our speedup. If we were to use any other attributes such as bid, name, zip,
            or state, those indexes would not result in a speedup if having a condition looking for the city.
    %2.3
        \subsubsection{Problem 2.3}

        \begin{center}
            \begin{tabular}{ || c | c || }
            \hline
                                            &   Execution Time  \\
            \hline
            \hline
            Before creating an index (A)    &   347.601 sec     \\
            \hline
            After  creating an index (B)    &     3.058 sec     \\
            \hline
            Time diference (A-B)            &   344.543 sec     \\
            \hline
            \end{tabular}
        \end{center}
        


\end{document}
%   Documents ends
